\documentclass[12pt]{article}

\usepackage[utf8]{inputenc}
\usepackage{listings}
\usepackage{color}
\usepackage{hyperref}
\usepackage{graphicx}
\usepackage{subfigure}
\usepackage{xparse}
\usepackage{amsmath}
\usepackage{amssymb}
\usepackage{mathtools}
\usepackage[font=small,labelfont=bf]{caption}
\usepackage{minted}
\usepackage{xcolor}

\definecolor{codegreen}{rgb}{0.1,0.5,0.1}
\definecolor{codegray}{rgb}{0.5,0.5,0.5}
\definecolor{codepurple}{rgb}{0.5,0,0.33}
\definecolor{backcolour}{rgb}{0.95,0.95,0.95}

\lstdefinestyle{mystyle}{
    backgroundcolor=\color{backcolour},   
    commentstyle=\color{codegreen},
    keywordstyle=\color{blue},
    numberstyle=\tiny\color{codegray},
    stringstyle=\color{codepurple},
    basicstyle=\ttfamily\footnotesize,
    breakatwhitespace=false,         
    breaklines=true,                 
    captionpos=b,                    
    keepspaces=true,                 
    numbers=left,                    
    numbersep=5pt,                  
    showspaces=false,                
    showstringspaces=false,
    showtabs=false,                  
    tabsize=2
}

\lstset{style=mystyle}
\usemintedstyle{colorful}
\title{Bachelor Thesis\\ Decoding the Color Code}
\author{Clemens Schumann, \\
Advised by Peter-Jan Derks}

\begin{document}
\maketitle
\begin{abstract}
    \begin{center}
The color code is an error-correcting code that 
uses three colorable three-regular Tanner graphs to perform quantum error correction.
In this thesis, we will give an overview of quantum error correction codes and introduce
decoding schemes for the color code, as well as compare their thresholding performance.
    \end{center}
\end{abstract}
\newpage
\tableofcontents
\newpage
\section{Introduction}
In the following we will give an overview of the utilized
algebra in quantum information theory.\\
To this end we must first return to the fundamentals of quantum
mechanics.
\subsection{Schroedinger picture}
In the Schroedinger picture, we focus on the time evolution of states:
\begin{equation}  
	|\psi\rangle = |\psi\rangle(t) 
\end{equation}
In this picture we can introduce quantum circuit diagram notation, whereby:
\begin{itemize}
	\item States progress in time along horizontal parallel lines
	\item Time goes from left to right
	\item Gates denoted X,Y,Z are the single qubit pauli operators
		$\sigma_x,\sigma_y,\sigma_z$
	\item Gates can act on one or multiple qubits, whereby an X gate 
		on qubit 1 in a 3-qubit system should be interpreted as:
		\\$(X\otimes \mathbb{I} \otimes \mathbb{I}) (|\psi_1\rangle
		\otimes |\psi_2\rangle \otimes |\psi_3\rangle)$
\end{itemize}
\begin{figure}[h!]
	\begin{center}
\includegraphics[scale=0.5]{img/cnotMeasureCircuit.png}\\
	\caption{A Quantum Circuit to implement a measurement based\\
		Controlled-$X_{|\psi\rangle_{control}\rightarrow |\psi\rangle_{target}}$ Gate,
		where $|0\rangle$ is the +1 eigenstate in $\sigma_{z}$-basis.}
	\label{fig:circuit1}
	\end{center}
\end{figure}
\newpage

As can be seen explicitly calculated in the familiar Schroedinger 
picture in Appendix~\ref{sec:calc1}, the circuit from figure~\ref{fig:circuit1}
implements a CNOT-gate from the control qubit to the target qubit.

We will now analyze this circuit in the Heisenberg picture,
finding that it results in an equal output.

\subsection{Heisenberg Picture and stabilizer Formalism}
\subsubsection{Stabilizer Group}
We call an operator/gate S, to which the input state is an 
eigenvector ($S|\psi\rangle=|\psi\rangle$), a ``stabilizer'' of that input state. \\
The ``stabilizer group'' is a generating subset of the set
of such operators.

We write these stabilizers as tensor-products of pauli operators
$P \in P_{G}$,
where pauli operators are the operators on $\mathbb{F}_{2}$ such that:

$\forall P\in P_{G}, n\in \mathbb{N}: P^{2n}=\mathbb{I}$.

In the Heisenberg picture, stabilizers are tracked instead of
states. 
The stabilizer group $S_{G}$ is the group generated by
the set of stabilizers:
\begin{equation}
	S_{G} = \langle S_{0},..,S_{n}\rangle: S|\psi_{in}\rangle = 
	|\psi_{in}\rangle \forall S \in S_{G}
\end{equation}

So for the example in figure~\ref{fig:circuit1} it is the group
of operators to whom
$|\psi_{control}\rangle \otimes |0\rangle \otimes 
|\psi_{target}\rangle$ is an eigenstate, namely 
$\mathbb{I}\otimes Z \otimes \mathbb{I}$ (and trivially
$\mathbb{I}\otimes\mathbb{I}\otimes\mathbb{I}$, which we choose
to ignore as a stabilizer since any three-qubit state
is stabilized by it).

The stabilizer group is always an abelian group and its elements 
 commute, since if:

\begin{equation}
	\label{abelian_stabilizers_equation}
	\forall A,B \in S: AB|\psi\rangle = BA|\psi\rangle = |\psi\rangle
	\Rightarrow [A,B]|\psi\rangle=0
\end{equation}

\subsubsection{Effect of Gates on stabilizers}
In order to determine the effect a Gate operation A has on a
stabilizer, consider the following:

If $S|\psi\rangle = |\psi\rangle$ then:
\begin{equation}
A|\psi\rangle = AS|\psi\rangle = AS\mathbb{I}|\psi\rangle
	= \underbrace{ASA^{\dagger}}_{=S'}A|\psi\rangle
\end{equation}
So we now know that the post-gate state is an Eigenstate of $S'$.

Therefore $S'_{G} = \langle AS_{0}A^{\dagger},...,AS_{n}A^{\dagger}\rangle$.
%This generator expression can be further reduced by taking out
%all $S' \in S'_{G}$ for which $\exists J' \in S'_{G}: \{S',J'\}=0$

\subsubsection{Effect of measurements on stabilizers}
A pauli measurement operator M can either commute with all stabilizer
operators, in which case M itself is a stabilizer already and the
measurement has no effect on the state, or anticommute with at
least one operator in $S_{G}$, since pauli operators as well as
their tensor-products can only commute or anti-commute with each
other.

In that case, to obtain the new stabilizers  $S'_{G}$, proceed
as follows:
\begin{itemize}
	\item Identify $S\in S_{G}: {S,M}=0$
	\item Remove S from $S_G$
	\item Add $M$ to $S_G$ 
	\item for $N \in S_G \cup\overline{X}\cup\overline{Z}:$
		if ${N,M}==0: N=SN$ 
\end{itemize}
where $\overline{X}$ and $\overline{Z}$ are the sets of 
logical Xs and Zs respectively.

\subsubsection{Circuit Analysis in Stabilizer formalism}
After a measurement M, an n qubit input state will always 
collapse into either the +1 or the -1 eigenstate of the 
measurement operator.

In the first case the acting measurement operator becomes 
$\mathbb{I}^{\otimes n}+M$, in the second it becomes
$\mathbb{I}^{\otimes n}-M$. Therefore, in the circuit shown in 
figure~\ref{fig:circuit1}, the measurements are:
\begin{align}
	M_{1} &= \mathbb{I}^{\otimes 3} \pm \mathbb{I}\otimes X
	\otimes X \\
	M_{2} & = \mathbb{I}^{\otimes 3} \pm X \otimes X \otimes
	\mathbb{I} \\
	M_{3} &= \mathbb{I}^{\otimes 3} \pm \mathbb{I} \otimes X
	\otimes \mathbb{I}
\end{align}


\newpage
\subsection{The Clifford Gates}

It has been proven in *reference source* that operators that 
map a state stabilized by some member of the Pauli-Group to 
a state stabilized by another member of the Pauli Group can be simulated efficiently
on a classical computer. The Group of operators that satisfy
this condition is called the Clifford Group.

For the decoder we wish to implement in
this thesis, it therefore makes sense to focus on those first and
foremost, as applying corrective gates is a computationally/
experimentally expensive task that should be put off until the latest
possible moment, and the propagation of an error until then can be
simulated efficiently. 

The Clifford Group can be generated by:

\begin{itemize}
    \item The Hadamard-Gate $H$, which performs single qubit
    basis changes from eigenstates of X to eigenstates of Z 
    and vice-versa:

    $H|+\rangle = |0\rangle$, $H|0\rangle = |+\rangle$,
    $H|-\rangle = |1\rangle$, $H|1\rangle = |-\rangle$
    
    \item The Phase-Gate $P$, which performs single qubit sign
    flips on the state parts which are $|1\rangle$ in the 
    computational basis:

    $
    P (\alpha|0\rangle \pm \beta |1\rangle) =
    \alpha|0\rangle \mp \beta |1\rangle
    $

    \item The CNOT-Gate, which on a two qubit system performs 
    an X gate on the second qubit if the first qubit is 
    $|1\rangle$, so maps:

    $
    \alpha |00\rangle + \beta |01\rangle + 
    \delta |10\rangle + \gamma |11\rangle \\
    \mapsto
    \alpha |00\rangle + \beta |01\rangle +
    \gamma |10\rangle + \delta |11\rangle
    $
\end{itemize}
In the $\sigma_z$-basis their matrix representations are:
\begin{itemize}
    \item
    \begin{math}
    H = 
    \frac{1}{\sqrt{2}}\cdot\left( 
    \begin{array}{cc}
       1 & 1 \\
       1 & -1 \\
    \end{array}
    \right)\end{math} 
;
    \begin{math}
    P = 
    \left(\begin{array}{cc}
        1 & 0 \\
        0 & i \\
    \end{array}\right)
    \end{math}
    
    \item 
    \begin{math}
    CNOT = 
    \left(\begin{array}{cccc}
        1 & 0 & 0 & 0 \\
        0 & 1 & 0 & 0 \\
        0 & 0 & 0 & 1 \\
        0 & 0 & 1 & 0 \\
    \end{array}\right)
   \end{math} 
\end{itemize}

\newpage
\subsection{Error detection and correction}
\subsubsection{Error Model}
One of the big challenges of physically realising a quantum 
computer is its subjection to noise in the real world.
Unlike classical computers, the type of error is not limited
to a bitflip, as even single qubit states have a theoretically
infinite amount of differing states to it on a bloch sphere,
and therefore an infinite amount of types of errors can have
occured in the presence of noise such as thermal or electromagnetical
noise.

Fortunately, this noise can be modeled as a sum of pauli gates.
Any single qubit error operator matrix E can be written as:
\begin{equation}
    E =
    \left(
    \begin{array}{cc}
        a & b \\
        c & d \\
    \end{array}
    \right) = 
    \alpha \mathbb{I} + \beta X + \delta Y + \gamma Z
\end{equation}
With an apropriate choice of $\alpha, \beta, \gamma, \delta$.
In effect, this means that with probability $\alpha$, the effect of the
error $E|\psi\rangle$ will be $\mathbb{I}$; with probability $\beta$ its effect
will be X, and so on.

Therefore it is sufficient to determine which of these errors $\mathbb{I}$, 
X, Y, Z has occured, and we can apply the same operator again to return to the 
initial state.
Since an identity noise occuring is irrelevant to us, and XY as
well as ZY (anti-) commute, we need only detect for X and Z
errors occuring in order to detect any single qubit errors. 


\subsubsection{Repetition code}
In order to correct errors, they must first be detected.
From classical computer science there are well known existing
codes, such as the repetition code.
For this error code information is encoded by repeating the 
intended message some amount of times, and then decoding it
by performing a majority vote on the transmitted message.
In Quantum error correction, we speak of [[n,k,d]] stabilizer
codes if an encoding scheme allows for n physical qubits to 
encode k logical qubits to a distance d.

\begin{figure}[h!]
	\begin{center}
	\captionsetup{justification=centering,margin=2cm}
	\includegraphics[scale=0.2]{./img/bitflipSyndromeExtraction3Rep.png}\\
	\caption{Bitflip Syndrome extractor\\
        +1 measurement result on first Ancilla indicates a bitflip error
        on qubits 1 or 2, +1 result on second ancilla indicates 
		bitflip on second or third qubit}
	\label{fig: syndrome extractor}
	\end{center}
\end{figure}

A quantum equivalent of the 3-bit repetition code performed on
the message $|1\rangle$ is is the [[3,1,$\frac{1}{2}$]] repetition
code depicted in 
figure~\ref{fig: syndrome extractor}, including so-called
``Syndrome Extraction''. A Syndrome is a stabilizer that can be
measured in order to detect wether and where an error has occured
in a multi-qubit system. It is crucial that the 
measurement of such Syndromes occurs without harming the actual
quantum information stored in the ``data-qubits''. Therefore
two additional ``Ancilla-qubits'' (both initialized to 
$|0\rangle$) are attached to the circuit via CNOTs.
This circuit is stabilized by IZZ and ZZI, measured by ancilla 
1/2. The measurement result will therefore be a vector of length
two, with each entry either being +1 or -1. To simplify the 
algebra this will be changed to the binary representation of 0 
for +1 and 1 for -1. 

To represent the code, Stabilizers can be stacked together to
a so-called parity-check-matrix, which satisfies:
\begin{equation}
	M_{pc}\cdot \vec{v}_{error} = \vec{v}_{syndrome}
\end{equation}
So e.g. the parity check matrix for the $[3,1,\frac{1}{2}]$
repetition code would be:

\begin{equation}
	M_{pc3} = \left( 
	\begin{array}{ccc}
		1 & 1 & 0 \\
		0 & 1 & 1
	\end{array}
	\right)
\end{equation}
And the syndrome for an X error on the first qubit would be
$\left(\begin{array}{c}1\\0\end{array}\right)$.

This way of encoding information however leaves two notable
issues:

For one, it only detects bitflip, or pauli-X errors occuring on
the stored quantum information. While using Hadamard gates one
could trivially adapt this code to instead detect pauli-Z errors,
it is not possible to use linear codes like the repetition code
to $simultaneously$ detect pauli-X and pauli-Z errors occuring.

Secondly, it also assumes a noise model of a ``Noisy Channel'',
which is not compatible with the actually encountered errors in
real physical quantum computers.
\newpage
\subsubsection{2D codes}
Previous research in computer science 
provides a toolset for generating valid codes
from existing encoding schemes. 
Hypergraph product codes, introduced by Tillich and Z\'emor,
of two 
existing codes will always remain a valid detection code.
We can therefore form a hypergraph product code of two repetition
codes for X error detection and Z error detection respectively,
obtaining the [[$d^2$,1,d]]``Surface-Code'' which can detect up
to d of $both$ X and Z errors, and 
therefore any pauli error happening.

The parity check matrix of a hypergraph product code is generated
by the parity check matrices of two valid codes in the following
way:
\begin{equation}
	H = \left(\begin{array}{cc}
		M_{pcX} & 0 \\
		0 & M_{pcZ} \\
	\end{array}\right)
\end{equation}

PUT IN A NICE VISUALISATION OF THE SURFACE CODE.


\newpage
\section{Error detection and correction}
The concept of (classical) error-correcting codes (ECC) was
introduced by Claude Shannon in 1948 \cite{shannon}.
Fundamentally, an ECC encodes $logical$ information within
a large superset of basic information carriers.

In the case of a classical computer, this means encoding a
bitstring within a system containing more physical bits
than the length of the encoded message, with the goal of message transmission
being resilient to some bits being faulty or subject to interference (i.e. EM-interference).

Analogously, in the case of a quantum computer this means encoding a \emph{logical}
qubit within a system of multiple qubits, with a similar goal of resilience towards
errors caused by external influences.

In this chapter, we will give an overview of different quantum error correction codes,
starting with adaptations of classical codes.
\subsection{Classical codes}
The concept of (classical) error-correcting codes (ECC) was
introduced by Claude Shannon in 1948.
Fundamentally, an ECC encodes $logical$ information within
a large superset of basic information carriers.
In the case of a classical computer, this means encoding a
bitstring within a system containing more physical bits
than the length of the encoded message, with the goal of message transmission
being resilient to some bits being faulty or subject to interference (i.e. EM-interference).

Two such classical ECCs are the repetition and the ring code.
We call these codes linear because their graph representations are
linear, or one-dimensional. In Quantum error correction, we speak of $[[n,k,d]]$ stabilizer
codes if an encoding scheme allows for $n$ physical qubits to 
encode $k$ logical qubits to an error distance of $d$, i.e. $d$ 
arbitrary individual errors being corrigible.
In the following, I will refer to linear or classical codes as having a 
distance of $\frac{1}{2}$, to indicate that they do not protect against an
arbitrary single-qubit error, but only against flips in one specific eigenbasis.
\subsubsection{Repetition code}
For this error code information is encoded by repeating the 
intended message some amount of times, and then decoding it
by performing a majority vote on the transmitted message.


\begin{figure}[h!]
	\begin{center}
	\captionsetup{justification=centering,margin=2cm}
	\includegraphics[scale=0.2]{./img/figures/bitflipSyndromeExtraction3Rep.png}\\
	\caption{Bitflip Syndrome extractor for [[3,1,$\frac{1}{2}$]] repetition code\\
        +1 measurement result on first ancilla indicates a bitflip error
        on qubits 1 or 2, +1 result on second ancilla indicates 
		bitflip on second or third qubit}
	\label{fig: syndrome extractor}
	\end{center}
\end{figure}

A quantum equivalent of the 3-bit repetition code performed on
the message $|1\rangle$ is the [[3,1,$\frac{1}{2}$]] repetition
code depicted in 
figure~\ref{fig: syndrome extractor}
, including so-called
$syndrome\ extraction$. A syndrome is a stabilizer that can be
measured to detect whether and where an error has occurred
in a multi-qubit system. It is crucial that the 
measurement of such syndromes occurs without harming the actual
quantum information stored in the $data-qubits$. Therefore
two additional \emph{ancilla-qubits} (both initialized to 
$|0\rangle$) are attached to the circuit via CNOTs.
This circuit is stabilized by IZZ and ZZI, measured by ancilla 
1 and 2. The measurement result will therefore be a vector of length
two, with each entry either being +1 or -1. To simplify the 
algebra this will be changed to the binary representation of 0 
for +1 and 1 for -1. 

To represent the code, stabilizers can be stacked together to
a so-called parity-check-matrix, which satisfies:
\begin{equation}\label{eq: pcm}
	M_{pc}\cdot \vec{v}_{error} = \vec{v}_{syndrome}
\end{equation}
So e.g. the parity check matrix for the $[3,1,\frac{1}{2}]$
repetition code would be:

\begin{equation}
	M_{pc3} = \left( 
	\begin{array}{ccc}
		1 & 1 & 0 \\
		0 & 1 & 1
	\end{array}
	\right)
\end{equation}
And the syndrome for an X error on the first qubit would be
$\left(\begin{array}{c}1\\0\end{array}\right)$.

If we draw a graph to represent this code, with here square nodes being
 ancilla qubits
and round nodes being data qubits, we obtain the following:

\begin{figure}[h!]
	\begin{center}
	\captionsetup{justification=centering,margin=2cm}
	\includegraphics[scale=0.4]{./img/figures/rep_3_graph.png}\\
	\caption{Graph for [[3,1,$\frac{1}{2}$]] repetition code with error on
    node 1 marked in green and resulting syndrome marked red.
    Squares represent ancilla qubits and circles represent data qubits.}
        
	\label{fig: rep_graph}
	\end{center}
\end{figure}

\newpage
\subsubsection{Ring code}
The ring code's graph essentially simply loops around at the repetition
code's single-edged ancilla nodes via an additional ancilla. 
It's edge matrix where the \emph{n}th row
represents which data qubit is connected to the nth ancilla
qubit is the following:
\begin{equation}
    M_{pc3} = \left(
        \begin{array}{ccc}
            1 & 1 & 0\\
            0 & 1 & 1\\
            1 & 0 & 1\\
        \end{array}
        \right)
\end{equation}

\begin{figure}[h!]
	\begin{center}
	\captionsetup{justification=centering,margin=2cm}
	\includegraphics[scale=0.25]{./img/figures/ring_3_graph.png}\\
	\caption{Graph for [[3,1,$\frac{1}{2}$]] ring code with error on node
    1 marked in green and resulting syndrome marked in red.
    Squares represent ancilla qubits and circles represent data qubits.}
        
	\label{fig: ring_graph}
	\end{center}
\end{figure}


\newpage
\subsection{Quantum Error Model}
This way of encoding information however leaves a notable
issue:
It only detects bitflip, or Pauli-X, errors occurring on
the stored quantum information. While using Hadamard gates one
could trivially adapt this code to instead detect Pauli-Z errors,
it is not possible to use linear codes like the repetition code
to $simultaneously$ detect Pauli-X and Pauli-Z errors occurring.

Unlike classical computers, on a quantum computer the type of error
 is not limited
to a bitflip. Even for single-qubit states there exists an
infinite amount of differing possible errors, since when representing a single
qubit state as a vector on a Bloch sphere it immediately becomes apparent
that there are an infinite number of vectors on that sphere which are different
from it. It turns out though, that the change from one normalized
state to another is merely a sum of rotations.

Noise can therefore be modeled as a sum of Pauli gates.
Any single qubit error operator matrix E can be written as:
\begin{equation}
    E =
    \left(
    \begin{array}{cc}
        a & b \\
        c & d \\
    \end{array}
    \right) = 
    \alpha \mathbb{I} + \beta X + \delta Y + \gamma Z
\end{equation}
With an apropriate choice of $\alpha, \beta, \gamma, \delta$.
In effect, this means that with probability $\alpha$, the effect of the
error $E|\psi\rangle$ will be $\mathbb{I}$; with probability $\beta$ its effect
will be X, and so on.

It is hence sufficient to determine which of these errors $\mathbb{I}$, 
X, Y or Z has occurred, and we can apply the appropriate operator to return to the 
initial state.
Since an identity noise occurring is irrelevant to us, and XY as
well as ZY (anti-) commute, we need only detect for X and Z
errors occurring in order to detect any single qubit errors. 
(because of the commutation relation between \{X,Y\} and \{Y,Z\} a Y error
will appear as both an X and Z error).
\subsection{Topological codes}
Hypergraph product codes, introduced by Tillich and Z\'emor \cite{tillichzemor},
provide a toolset for generating valid codes
from existing encoding schemes.
A hypergraph product code of two
existing codes will always remain a valid detection code.

The parity check matrix $H$ of a hypergraph product code is generated
by two m by n parity check matrices of valid codes in the following
way:
\begin{equation}
	M_{PC_{Hypergraph}} = \left(\begin{array}{cc}
		\left(M_{pc1} \otimes \mathbb{I}_{n_2}| 
        \mathbb{I}_{m_1} \otimes M_{pc2}^T \right) & 0 \\
		0 & \left(\mathbb{I}_{n_1}\otimes 
        M_{pc2} | M_{pc1}^T \otimes \mathbb{I}_{m_2}\right)\\
	\end{array}\right)
\end{equation}

\subsubsection{Surface code}
We can therefore form a hypergraph product code of two repetition
codes to
obtain the [[$d^2$,1,d]] ``Surface-Code'' which can detect up
to d of $both$ X and Z errors, and 
therefore any error happening \cite{joschka}.
We can draw this code as a graph, whereby the code's stabilizers
are understood as an adjacency Matrix of data to ancilla qubits.
Unlike in figure \ref{fig: surface_code}, it is possible to draw this 
graph without colors, by drawing it such that the data qubits are on 
edges of the graph and the ancilla qubits for Z-checks are on faces
while the ancilla qubits for X-checks lie on nodes.\\
Like the repetition code, the Surface code is a code that is regular until
its boundary nodes. 

\begin{figure}[h!]
	\begin{center}
	\captionsetup{justification=centering,margin=2cm}
	\includegraphics[scale=0.35]{./img/figures/d5surfaceCode.png}\\
	\caption{Distance 5 Surface code with data qubits in white and 
    ancilla qubits in grey. Green Faces tepresent X stabilizers
    and Red faces represent Z stabilizers.
    Errors on data qubits are marked
    by respective Pauli names and violated stabilizers are marked in red.}
	\label{fig: surface_code}
	\end{center}
\end{figure}


INCLUDE LOGICAL OPERATORS
INSERT FIGURE SHOWING SURFACE CODE
\newpage

\subsubsection{Toric code}
Similarly, a hypergraph product code of two ring codes can be 
generated. We call this code the "Toric code".\\
\begin{figure}[h!]
	\begin{center}
	\captionsetup{justification=centering,margin=2cm}
	\includegraphics[scale=0.4]{./img/figures/toric_5_graph.png}\\
	\caption{Graph for [[49,1,7]] toric code}
        
	\label{fig: toric_graph}
	\end{center}
\end{figure}
Notably, this resembles a donut, or torus.
The logical operators on the toric code are loops, so a circle of 
'errors' on nodes is a logical X operator, and a circle of 'errors'
on faces is a logical Z operator.
\newpage

\subsubsection{Color code}
The color code's parity-check-matrix's 
rows are both the code's X stabilizers and Z stabilizers.
Any three-colorable graph represents a valid color code.
On the color code, an error is bounded by syndromic faces of all colors.
\\
\begin{figure}[h!]
	\begin{center}
	\captionsetup{justification=centering,margin=2cm}
	\includegraphics[scale=0.6]{./img/figures/steane.png}\\
	\caption{Graph for [[7,1,3]] color code, also known as the
    Steane code \cite{steane} , and its stabilizers}
        
	\label{fig: color_graph}
	\end{center}
\end{figure}
% habe ich von erster google bildsuchen seite

\newpage
\newpage
\section{Decoding Schemes}
\subsection{Decoders for Surface/Toric codes}
The Syndromes on the surface/toric code are a set of nodes and
faces on the code graph. The node ancilla syndromes correspond to
Z errors, while the face ancilla syndromes correspond to X errors.
Since neighboring errors will trigger an ancilla that is between 
both errors twice, a chain of errors will only appear as two ancilla
syndrome bits being flipped at its borders.
The task of a decoding scheme for a surface/toric code is thus to
find the shortest paths between node pairs/face pairs, since the most likely
chain of errors to occur given a $<50\%$ physical error rate is the 
shortest one.\\
In practice, decoders for surface/toric codes only need to be able to
match nodes, since the matching of faces is just matching nodes on the 
dual graphs and the resulting data qubit errors can just be joined
(i.e. if an edge is found to have an error on both the X graph as well as the
dual Z graph, we know a Y error has occurred on that edge/data-qubit).
An example of a distance 5 surface code with two Z errors, one X error and
one Y error is shown in figure \ref{fig: surface_code}.
As with the ringcode, the decoding problem can be seen as either the
solution of equation \ref{eq: pcm} for a minimum weight $\vec{v}_{error}$
or as a graph matching problem.
\subsubsection{MWPM decoding}
\begin{enumerate}
    \item Find a set of unmatched nodes that can be reached from the 
    matching by alternating between matched and unmatched edges. 
    Call these nodes "augmenting nodes".
    \item Find an augmenting path starting from each augmenting node,
    i.e. a path that starts and ends with an unmatched node, and
    alternates between matched and unmatched edges. 
    \item If such a path is found, flip all edges along it from matched
    to unmatched, and vice versa.
    \item Repeat until no augmenting path is found.
\end{enumerate}
This decoding scheme has the advantage of being guaranteed
to find a global optimum of decoding edge paths, i.e. it finds the shortest vector
of edges that are bounded by the syndrome nodes.
Under the assumption of high error rates and/or large decoding 
graphs, this scheme also requires significantly less
computational memory overhead than the union-find scheme
\cite{MWPMDecoder}.
\subsubsection{Union Find decoder}
\begin{enumerate}
    \item Initialize a cluster set for each syndrome node
    \item Grow each cluster by one edge in each direction
    \item Merge all clusters that share a node
    \item For all clusters with an even amount of syndrome nodes,
    perform MWPM within that cluster. Pop the found error edges from
    the graph.
    \item Repeat until all clusters are merged/discarded.
\end{enumerate}
While the union-find decoder is faster for small to medium
sized graphs and relatively simple to implement,
 it is not guaranteed to find a global optimum
and its performance degrades significantly for large
graphs and high error rates \cite{UFDecoder}.
For this reason, a MWPM algorithm was chosen for decoding the toric
subgraphs of the color code in our lifting decoder thresholding
in chapter \ref{sec: lifting}.

\subsection{Color code decoders}
Unlike the surface and toric codes, in the color code the 
data qubits sit on the graphs nodes, and the ancillas on the 
graphs faces. Decoding the color code entails matching 
three differently colored faces to its enclosed nodes.
This is a significantly more challenging task than
decoding the 2D-codes, since three-colored graph matching is a confirmed
NP-hard problem. (reference delfosse paper)
\subsubsection{Lookup table decoding}
This takes a long time to generate and cannot be done to scale,
but is a nice toy.
\subsubsection{Lifting decoder}\label{sec: lifting}
The Lifting decoder works as follows:
\begin{itemize}
    \item Create dual of graph
    \item Generate single-edge-colored subgraphs of the dual
    \item Decode subgraphs using MWPM/Union-Find
    \item Unify all edges from subgraph corrections
    \item Find all shortest-length loops on this union
    \item NOLIFTING CASE ????
    \item All nodes bounded by the faces that are elements of the shortest-length loop sets
    are error nodes. 
\end{itemize}
The reason for this seemingly complicated procedure is that, on its
own, 3-colored matching would be an NP-hard problem. However, by sub-tiling
the graph into smaller subgraphs, we can reduce the problem of decoding
e.g. a honeycomb lattice toric color code to a set of
MWPM-decodable toric graphs that merely need to be "lifted" into a 
combination of subgraph decodings to decode the original color code
graph \cite{delfosse}. 
The polynomial time complexity of the lifting decoder does not
violate the NP-hardness of the 3-color matching problem, since the 
lifting procedure can also fail (NOLIFTING case).
\newpage
\subsection{Thresholds}
To compare different codes and decoding schemes
we introduce the concept of thresholds, whereby the threshold
of a specific code of scalable distance with a specific decoding 
scheme is defined as the physical error rate $per$ at which the logical
error rate becomes greater than $50\%$ in the limit of infinite 
distance. \\
Thresholds can vary depending on the error model, i.e. 
some codes can have a higher threshold for X than for Z errors.
For simplicity's sake in the following, we will assume equal 
X, Y and Z error rates of $\frac{per}{3}$.

Using this error model, we found a threshold of $x\pm y \%$ and
$z\pm q\%$ for the surface and toric code respectively. 
(A BUNCH OF FIGURES AND APPENDIX REFS)

On the color code, we found a threshold of $a\pm b\%$ using the 
lookup table decoder on the Steane code and $c\pm b\%$ using the 
lifting decoder for the scalable hexagonal toric color code. 
(Reference appendix code, include a figure)
\section{Thresholds}
To compare different codes and decoding schemes
we introduce the concept of thresholds, whereby the threshold
of a specific code of scalable distance with a specific decoding 
scheme is defined as the physical error rate $per$ at which the logical
error rate becomes greater than $50\%$ in the limit of infinite 
distance. \\
Thresholds can vary depending on the error model, i.e. 
some codes can have a higher threshold for X than for Z errors.
For simplicity's sake in the following, we will assume equal 
X, Y and Z error rates of $\frac{per}{3}$.

Using this error model, we found a threshold of $16.3\pm 0.5 \%$ for the surface code,
 $16.0\pm 0.5\%$ for the toric code and $16.1\pm0.5$ from
subfigures b), d) and f) in Figure \ref{fig:surface_threshold}.
Their thresholds are within single error margins of each other, and 
can therefore be called identical.

Since the Steane code for which we generated a lookup table is not 
a distance-scalable code, only a $pseudo$-threshold can
be found here, i.e. the crossing point to worse performance than unencoded
information. As can be seen in Figure \ref{fig: steane_threshold}, the
pseudo-threshold lies around $(1 \pm 0.5)10^{-5}$.

\begin{figure}[h!]
	\centering
	\captionsetup{justification=centering,margin=2cm}
    \subfigure[Lookup table Steane code threshold]{\includegraphics[width=0.5\textwidth]{./img/figures/thresholds/steaneLookupThreshold.png}}\hfill
	\subfigure[Detailed view at around $per=10^{-5}$]{\includegraphics[width=0.5\textwidth]{./img/figures/thresholds/steaneLookupThresholdPrecise.png}}\hfill
	\caption{Lookup table pseudo threshold for the Steane code, generating code can be found in Appendix
    \ref{App: steane_thresholding}}
        
	\label{fig: steane_threshold}
\end{figure}

A prototype for a scalable lifting decoder for the hexagonal toric color code 
that only corrects separate individual errors was  implemented in \ref{App: lifting}.
\begin{figure}[H]
    \centering
    \subfigure[Surface code MWPM thresholding overview]{\includegraphics[width=0.5\textwidth]{img/figures/thresholds/surfaceThresholdOverview.png}}\hfill
    \subfigure[Detailed view for precise threshold determination of surface code]{\includegraphics[width=0.5\textwidth]{img/figures/thresholds/surfaceThresholdVeryPrecise.png}}
    \subfigure[Toric code MWPM thresholding overview]{\includegraphics[width=0.5\textwidth]{img/figures/thresholds/toricThresholdOverview.png}}\hfill
    \subfigure[Detailed view for precise threshold determination of toric code]{\includegraphics[width=0.5\textwidth]{img/figures/thresholds/toricThresholdVeryPrecise.png}}
    \subfigure[Cylinder code MWPM thresholding overview]{\includegraphics[width=0.5\textwidth]{img/figures/thresholds/cylinderThresholdOverview.png}}\hfill
    \subfigure[Detailed view for precise threshold determination of cylinder code]{\includegraphics[width=0.5\textwidth]{img/figures/thresholds/cylinderThresholdVeryPrecise.png}}
    \caption{Thresholding of the surface/toric/cylinder code using the MWPM decoder implemented in the PyMatching \cite{MWPMDecoder} library.
    Generating code can be found in Appendix \ref{App: surface_thresholding}}
    \label{fig:surface_threshold}
\end{figure}



% $c\pm b\%$ using the 
% lifting decoder for the scalable hexagonal toric color code. 
% (Reference appendix code, include a figure)
\newpage
\section{Conclusion}
In this thesis, we gave an overview of existing
quantum codes as well as some decoding schemes.
The determined thresholds of $16.3\pm 0.5 \%$ for the surface code
and $16.0\pm0.5\%$ for the toric code were within single and threefold error margin
respectively to the literature values\cite{surfaceToricThreshold}. 
Their thresholds were however not distinguishable with great confidence, 
and especially for the cylindric code it might be of interest to 
calculate these thresholds more precisely by using more significant computational
resources in future works.
The pseudo-threshold for the Steane code was found to be around $10^{-5}$, which is
the same as in the literature\cite{steaneThreshold}. 
While the lifting decoder for the hexagonal toric lattice color code did not produce
thresholdable output, it did work as a proof-of-concept on smaller error vectors as in
\ref{fig: lifting}.
Future work could include adapting a better cycle-finder algorithm for the lifted 
subgraph.
\newpage
\section{Appendix}
\subsection{Calculation 1}
\label{sec:calc1}

In the quantum circuit depicted in figure \ref{fig:circuit1} the input
state can be written as $|\psi_{control}\rangle \otimes |0\rangle 
\otimes |\psi_{target}\rangle$ and the measurement in the first 
timestep can be expressed as $\mathbb{I}\otimes X \otimes X$.\\
\\
The initial state $|\phi_{t=0}\rangle$ = $|\psi_{control}\rangle \otimes |\psi_{ancilla}\rangle \otimes
|\psi_{target}\rangle$\\
where\\
$|\psi_{control}\rangle = \alpha|0\rangle+\beta|1\rangle$
\\
$|\psi_{ancilla}\rangle = |0\rangle$
\\
$|\psi_{target}\rangle = \gamma|0\rangle+\delta|1\rangle$
\\
therefore:
\begin{equation}
|\phi_{t=0}\rangle = \alpha \left( \gamma |000\rangle + \delta |001\rangle\right)+
\beta \left( \gamma |100\rangle + \delta |101\rangle \right)
\end{equation}

If the first measurement result is +1, the state becomes:
\begin{align*}
	|\phi^{+}_{t=1}\rangle 
	&= \frac{1}{2}\left(\mathbb{I}\otimes\mathbb{I}\otimes\mathbb{I}+
	\mathbb{I}\otimes X \otimes X\right)|\phi_{t=0}\rangle\\
	&= \alpha \left( 
	\gamma \left( |000\rangle + |011\rangle \right) +
	\delta \left( |001\rangle + |010\rangle \right) \right) \\
	&+ \beta \left(
	\gamma \left( |100\rangle + |111\rangle \right) +
	\delta \left( |101\rangle + |110\rangle \right) \right)
\end{align*}
if the result is -1, it becomes:
\begin{align*}
	|\phi^{-}_{t=1}\rangle 
	&= \frac{1}{2}\left(\mathbb{I}\otimes\mathbb{I}\otimes\mathbb{I}-
	\mathbb{I}\otimes X \otimes X\right)|\phi_{t=0}\rangle\\
	&= \alpha \left( 
	\gamma \left( |000\rangle - |011\rangle \right) +
	\delta \left( |001\rangle - |010\rangle \right) \right) \\
	&+ \beta \left(
	\gamma \left( |100\rangle - |111\rangle \right) +
	\delta \left( |101\rangle - |110\rangle \right) \right)
\end{align*}

In the case of the +1 Measurement $\rightarrow$ a=0:
\begin{align*}
	|\phi^{++}_{t=2}\rangle 
	&= \frac{1}{2}\left(\mathbb{I}\otimes\mathbb{I}\otimes\mathbb{I}+
	Z\otimes Z \otimes \mathbb{I}\right)|\phi^{+}_{t=1}\rangle\\
	&= (|000\rangle\langle000| +|001\rangle\langle001| +
	|110\rangle\langle110| +|111\rangle\langle111|)|\phi^{+}_{t=1}\rangle\\
	&= \alpha \left( \gamma |000\rangle + \delta |001\rangle \right)
	+ \beta \left( \gamma |111\rangle + \delta |110\rangle\right) 
\end{align*}
\begin{align*}
	|\phi^{+-}_{t=2}\rangle 
	&= \frac{1}{2}\left(\mathbb{I}\otimes\mathbb{I}\otimes\mathbb{I}-
	Z \otimes Z \otimes \mathbb{I}\right)|\phi^{+}_{t=1}\rangle\\
	&= (|010\rangle\langle010| +|011\rangle\langle011| +
	|100\rangle\langle100| +|101\rangle\langle101|)|\phi^{+}_{t=1}\rangle\\
	&= \alpha \left( \gamma |011\rangle + \delta |010\rangle \right)
	+ \beta \left( \gamma |100\rangle + \delta |101\rangle \right) 
\end{align*}
In the case of the -1 Measurement $\rightarrow$ a=1:
\begin{align*}
	|\phi^{-+}_{t=2}\rangle 
	&= \frac{1}{2}\left(\mathbb{I}\otimes\mathbb{I}\otimes\mathbb{I}+
	Z\otimes Z \otimes \mathbb{I}\right)|\phi^{-}_{t=1}\rangle\\
	&= \alpha \left( \gamma |000\rangle + \delta |001\rangle \right)
	- \beta \left( \gamma |111\rangle + \delta |110\rangle\right) 
\end{align*}
\begin{align*}
	|\phi^{--}_{t=2}\rangle 
	&= \frac{1}{2}\left(\mathbb{I}\otimes\mathbb{I}\otimes\mathbb{I}-
	Z\otimes Z \otimes \mathbb{I}\right)|\phi^{-}_{t=1}\rangle\\
	&= - \alpha \left( \gamma |011\rangle + \delta |010\rangle \right)
	+ \beta \left( \gamma |100\rangle + \delta |101\rangle \right) 
\end{align*}
Now the applied measurement is 
$\mathbb{I} \otimes X \otimes \mathbb{I}$, which means:
\begin{align*}
	|\phi^{+++}_{t=3}\rangle 
	&= \frac{1}{2}\left(\mathbb{I}\otimes\mathbb{I}\otimes\mathbb{I}+
	\mathbb{I}\otimes X \otimes \mathbb{I}\right)|\phi^{++}_{t=2}\rangle\\
	&= \frac{1}{2}((|010\rangle + |000\rangle)\langle000|
	+ (|011\rangle + |001\rangle)\langle001|\\
	&+ (|000\rangle + |010\rangle)\langle010|
	+ (|001\rangle + |011\rangle)\langle011|\\
	&+ (|110\rangle + |100\rangle)\langle100|
	+ (|111\rangle + |101\rangle)\langle101|\\
	&+ (|100\rangle + |110\rangle)\langle110|
	+ (|101\rangle + |111\rangle)\langle111|)
	|\phi^{++}_{t=2}\rangle\\
	&= \frac{1}{2}(\alpha \left( 
	\gamma (|000\rangle + |010\rangle)+
	\delta (|011\rangle + |001\rangle) \right) \\
	&+ \beta \left( 
	\gamma (|101\rangle + |111\rangle)+
	\delta (|100\rangle + |110\rangle) \right))
\end{align*}
In this case, a,b and c would each be zero, therefore no further gate would be applied.\\
As intended, this state is equivalent to 
$CNOT_{|\psi_{Control}\rangle\rightarrow |\psi_{Target}\rangle} |\phi_{t=0}\rangle$.
\\
If the last measurement result is -1:
\begin{align*}
	|\phi^{++-}_{t=3}\rangle 
	&= \frac{1}{2}\left(\mathbb{I}\otimes\mathbb{I}\otimes\mathbb{I}-
	\mathbb{I}\otimes X \otimes \mathbb{I}\right)|\phi^{++}_{t=2}\rangle\\
	&= \frac{1}{2}((|010\rangle + |000\rangle)\langle000|
	+ (|001\rangle-|011\rangle )\langle001|\\
	&+ (|010\rangle-|000\rangle)\langle010|
	+ (|011\rangle-|001\rangle)\langle011|\\
	&+ (|100\rangle-|110\rangle)\langle100|
	+ (|101\rangle-|111\rangle)\langle101|\\
	&+ (|110\rangle-|100\rangle)\langle110|
	+ (|111\rangle-|101\rangle)\langle111|)
	|\phi^{++}_{t=2}\rangle\\
	&= \frac{1}{2}(\alpha \left( 
	TODOTODOTODOTODOTODO
	\gamma (|000\rangle + |010\rangle)+
	\delta (|011\rangle + |001\rangle) \right) \\
	&+ \beta \left( 
	\gamma (|101\rangle + |111\rangle)+
	\delta (|100\rangle + |110\rangle) \right))
\end{align*}
Notably, each measurement sequence has a differing resulting ancilla state, 
however we do not care since ancillas are meant to be discarded.
\\
For now, the other 7 final computation steps are left as an exercise
to the reader, however I probably will still finish that.

\subsection{Lookup table decoding}
\subsubsection{Table generation}\label{App: lookup_table}
\lstinputlisting[language=python]{lib/betterlookup.py}

\subsubsection{Thresholding}\label{App: steane_thresholding}
\lstinputlisting[language=Python]{lib/lookupthreshold.py}


\subsection{Lifting Decoder}\label{App: lifting}
\lstinputlisting[language=python]{lib/hexcolor.py}
\subsection{Thresholds}
\subsubsection{Surface/Toric code thresholds}\label{App: surface_thresholding}
\lstinputlisting[language=python]{lib/surfaceToricThresholds.py}
\subsubsection{Color code thresholds}\label{App: color_thresholding}
\lstinputlisting[language=python]{lib/lookupthreshold.py}
\newpage
\bibliographystyle{IEEEtran}
\bibliography{chapterstex/references.bib}
\end{document}