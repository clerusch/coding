The study of Quantum Error Correction (QEC) is essential to the development
of quantum computers, as it provides a way to protect quantum
information from errors that can occur in 
a real-world setting subject to electromagnetic/thermal and other noise.

In this thesis, we will give an overview of the quantum error correction codes 
and introduce decoding schemes for the color code, a QEC code that uses
three colorable three-regular graph configurations of stabilizers
to perform quantum error correction.

We also compare the thresholding performance of various ECC codes and decoding 
schemes, finding a pseudo-threshold of $10^{-3}\%$ for the Steane color code using a 
lookup table decoder and around 16\% for the MWPM Surface/Toric/Cylindric codes.
While unable to determine these Thresholds more precisely due to computational
limitations, the author believes that upon further calculation the Cylindric code
could be found to have a threshold that lies between the higher surface code threshold
and the lower toric code threshold.
An attempt was made at constructing a lifting decoder for a toric hexagonal 
honeycomb lattice color code and a step-by-step guide to this construction is 
included, however this decoder is incomplete and only works for a small subset of 
possible errors (individually occurring ones) due to a bug in 
the lifting procedure and is therefore 
not included in the thresholding comparison.
