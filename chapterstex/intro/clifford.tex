In classical computation, a \emph{complete logical signature} is a group of
operators, which can be successively applied to express any general boolean 
computation. One example of such a signature is $\{\neg, \land\}$. 
A quantum equivalent of this is the Pauli Group amended by the 
\emph{Clifford group}, 
whereby the Clifford group is the group of operators that project eigenstates
of a Pauli group operator onto an eigenstate of a Pauli group operator.

While not enabling universal computation (e.g. the phase 
estimation in Shor's algorithm \cite{shor} would require an additional T gate),
the union of Clifford and Pauli group \emph{is} a complete logical signature for those quantum
operations that can be simulated efficiently on a classical computer
\cite{gottesmanFaultTolerant}.
This is relevant for quantum error correction, as applying corrective gates after
an error is computationally and experimentally expensive and should therefore be put
off until the first non-Clifford gate is encountered in the program.
Until that point the propagation of the error through the circuit can be simulated
efficiently.

The Clifford Group can be generated by:

\begin{itemize}
    \item The Hadamard-Gate $H$, which performs single qubit
    basis changes from eigenstates of X to eigenstates of Z 
    and vice-versa:

    $H|+\rangle = |0\rangle$, $H|0\rangle = |+\rangle$,
    $H|-\rangle = |1\rangle$, $H|1\rangle = |-\rangle$
    
    \item The Phase-Gate $P$, which performs single qubit sign
    flips on the state parts which are $|1\rangle$ in the 
    computational basis:

    $
    P (\alpha|0\rangle \pm \beta |1\rangle) =
    \alpha|0\rangle \mp \beta |1\rangle
    $

    \item The CNOT-Gate, which on a two qubit system performs 
    an X gate on the second qubit if the first qubit is 
    $|1\rangle$, so maps:

    $
    \alpha |00\rangle + \beta |01\rangle + 
    \delta |10\rangle + \gamma |11\rangle \\
    \mapsto
    \alpha |00\rangle + \beta |01\rangle +
    \gamma |10\rangle + \delta |11\rangle
    $
\end{itemize}
In the $\sigma_z$-basis their matrix representations are:
\begin{itemize}
    \item
    \begin{math}
    H = 
    \frac{1}{\sqrt{2}}\cdot\left( 
    \begin{array}{cc}
       1 & 1 \\
       1 & -1 \\
    \end{array}
    \right)\end{math} 
;
    \begin{math}
    P = 
    \left(\begin{array}{cc}
        1 & 0 \\
        0 & i \\
    \end{array}\right)
    \end{math}
    
    \item 
    \begin{math}
    CNOT = 
    \left(\begin{array}{cccc}
        1 & 0 & 0 & 0 \\
        0 & 1 & 0 & 0 \\
        0 & 0 & 0 & 1 \\
        0 & 0 & 1 & 0 \\
    \end{array}\right)
   \end{math} 
\end{itemize}
