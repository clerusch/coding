\subsection{The Clifford Gates}

It has been proven in *reference source* that operators that 
map a state stabilized by some member of the Pauli-Group to 
a state stabilized by another member of the Pauli-Group can be simulated efficiently
on a classical computer. The Group of operators that satisfy
this condition is called the Clifford-Group.

For the decoder we wish to implement in
this thesis it therefore makes sense to focus on those first and
foremost, as applying corrective gates is a computationally/
experimentally expensive task that should be put off to the latest
possible moment, and the propagation of an error until then can be
simulated efficiently. 

The Clifford-Group can be generated by:

\begin{itemize}
    \item The Hadamard-Gate $H$, which performs single qubit
    basis changes from eigenstates of X to eigenstates of Z 
    and vice-versa:

    $H|+\rangle = |0\rangle$, $H|0\rangle = |+\rangle$,
    $H|-\rangle = |1\rangle$, $H|1\rangle = |-\rangle$
    
    \item The Phase-Gate $P$, which performs single qubit sign
    flips on the state parts which are $|1\rangle$ in the 
    computational basis:

    $
    P (\alpha|0\rangle \pm \beta |1\rangle) =
    \alpha|0\rangle \mp \beta |1\rangle
    $

    \item The CNOT-Gate, which on a two qubit system performs 
    an X gate on the second qubit if the first qubit is 
    $|1\rangle$, so maps:

    $
    \alpha |00\rangle + \beta |01\rangle + 
    \delta |10\rangle + \gamma |11\rangle \\
    \mapsto
    \alpha |00\rangle + \beta |01\rangle +
    \gamma |10\rangle + \delta |11\rangle
    $
\end{itemize}
In the $\sigma_z$-basis their matrix representations are:
\begin{itemize}
    \item
    \begin{math}
    H = 
    \frac{1}{\sqrt{2}}\cdot\left( 
    \begin{array}{cc}
       1 & 1 \\
       1 & -1 \\
    \end{array}
    \right)\end{math} 
;
    \begin{math}
    P = 
    \left(\begin{array}{cc}
        1 & 0 \\
        0 & i \\
    \end{array}\right)
    \end{math}
    
    \item 
    \begin{math}
    CNOT = 
    \left(\begin{array}{cccc}
        1 & 0 & 0 & 0 \\
        0 & 1 & 0 & 0 \\
        0 & 0 & 0 & 1 \\
        0 & 0 & 1 & 0 \\
    \end{array}\right)
   \end{math} 
\end{itemize}
