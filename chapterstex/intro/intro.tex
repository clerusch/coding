A quantum computer operates on so-called $qudits$, which can be
any multi-level quantum system. 
Physical implementations of these include particles with 
spin, as well as controlled EM waves, i.e. lasers. \\
In this thesis, we will
focus on \emph{qubit}-based systems, i.e. two-level quantum systems as base 
units of computation.

In this chapter, we will analyze a quantum circuit diagram using different
pictures of quantum mechanics, namely the Schroedinger and the Heisenberg picture.
A quantum circuit diagram is a visual representation of the computation done
in a quantum computer, whereby:
\begin{itemize}
	\item States progress in time along horizontal parallel lines
	\item Time goes from left to right
	\item what kind of gates?
	\item Gates denoted X, Y, Z are the single qubit Pauli operators
		$\sigma_x,\sigma_y,\sigma_z$
    \item $M_{\{X,Y,Z\}^n}$ denotes an n qubit measurement of \{X,Y,Z\}
	\item Gates can act on one or multiple qubits, whereby an X gate 
		on qubit 1 in a 3-qubit system should be interpreted as
		\\$(X\otimes \mathbb{I} \otimes \mathbb{I}) |\psi_{1,2,3}\rangle$
\end{itemize}

\subsection{The Clifford Gates}

It has been proven in *reference source* that operators that 
map a state stabilized by some member of the Pauli-Group to 
a state stabilized by another member of the Pauli Group can be simulated efficiently
on a classical computer. The Group of operators that satisfy
this condition is called the Clifford Group.

For the decoder we wish to implement in
this thesis, it therefore makes sense to focus on those first and
foremost, as applying corrective gates is a computationally/
experimentally expensive task that should be put off until the latest
possible moment, and the propagation of an error until then can be
simulated efficiently. 

The Clifford Group can be generated by:

\begin{itemize}
    \item The Hadamard-Gate $H$, which performs single qubit
    basis changes from eigenstates of X to eigenstates of Z 
    and vice-versa:

    $H|+\rangle = |0\rangle$, $H|0\rangle = |+\rangle$,
    $H|-\rangle = |1\rangle$, $H|1\rangle = |-\rangle$
    
    \item The Phase-Gate $P$, which performs single qubit sign
    flips on the state parts which are $|1\rangle$ in the 
    computational basis:

    $
    P (\alpha|0\rangle \pm \beta |1\rangle) =
    \alpha|0\rangle \mp \beta |1\rangle
    $

    \item The CNOT-Gate, which on a two qubit system performs 
    an X gate on the second qubit if the first qubit is 
    $|1\rangle$, so maps:

    $
    \alpha |00\rangle + \beta |01\rangle + 
    \delta |10\rangle + \gamma |11\rangle \\
    \mapsto
    \alpha |00\rangle + \beta |01\rangle +
    \gamma |10\rangle + \delta |11\rangle
    $
\end{itemize}
In the $\sigma_z$-basis their matrix representations are:
\begin{itemize}
    \item
    \begin{math}
    H = 
    \frac{1}{\sqrt{2}}\cdot\left( 
    \begin{array}{cc}
       1 & 1 \\
       1 & -1 \\
    \end{array}
    \right)\end{math} 
;
    \begin{math}
    P = 
    \left(\begin{array}{cc}
        1 & 0 \\
        0 & i \\
    \end{array}\right)
    \end{math}
    
    \item 
    \begin{math}
    CNOT = 
    \left(\begin{array}{cccc}
        1 & 0 & 0 & 0 \\
        0 & 1 & 0 & 0 \\
        0 & 0 & 0 & 1 \\
        0 & 0 & 1 & 0 \\
    \end{array}\right)
   \end{math} 
\end{itemize}


\subsection{Schroedinger picture}
In the Schroedinger picture, we focus on the time evolution of states:
\begin{equation}  
	|\psi\rangle = |\psi\rangle(t) 
\end{equation}
In this picture we can introduce quantum circuit diagram notation, whereby:
\begin{itemize}
	\item States progress in time along horizontal parallel lines
	\item Time goes from left to right
	\item Gates denoted X,Y,Z are the single qubit pauli operators
		$\sigma_x,\sigma_y,\sigma_z$
	\item Gates can act on one or multiple qubits, whereby an X gate 
		on qubit 1 in a 3-qubit system should be interpreted as:
		\\$(X\otimes \mathbb{I} \otimes \mathbb{I}) (|\psi_1\rangle
		\otimes |\psi_2\rangle \otimes |\psi_3\rangle)$
\end{itemize}
\begin{figure}[h!]
	\begin{center}
\includegraphics[scale=0.5]{img/cnotMeasureCircuit.png}\\
	\caption{A Quantum Circuit to implement a measurement based\\
		Controlled-$X_{|\psi\rangle_{control}\rightarrow |\psi\rangle_{target}}$ Gate,
		where $|0\rangle$ is the +1 eigenstate in $\sigma_{z}$-basis.}
	\label{fig:circuit1}
	\end{center}
\end{figure}
\newpage

As can be seen explicitly calculated in the familiar Schroedinger 
picture in Appendix~\ref{sec:calc1}, the circuit from figure~\ref{fig:circuit1}
implements a CNOT-gate from the control qubit to the target qubit.

We will now analyze this circuit in the Heisenberg picture,
finding that it results in an equal output.

\subsection{Heisenberg Picture and stabilizer Formalism}
\subsubsection{Stabilizer Group}
We call an operator/gate S, to which the input state is an 
eigenvector ($S|\psi\rangle=|\psi\rangle$), a ``stabilizer'' of that input state. \\
The ``stabilizer group'' is a generating subset of the set
of such operators.

We write these stabilizers as tensor-products of pauli operators
$P \in P_{G}$,
where pauli operators are the operators on $\mathbb{F}_{2}$ such that:

$\forall P\in P_{G}, n\in \mathbb{N}: P^{2n}=\mathbb{I}$.

In the Heisenberg picture, stabilizers are tracked instead of
states. 
The stabilizer group $S_{G}$ is the group generated by
the set of stabilizers:
\begin{equation}
	S_{G} = \langle S_{0},..,S_{n}\rangle: S|\psi_{in}\rangle = 
	|\psi_{in}\rangle \forall S \in S_{G}
\end{equation}

So for the example in figure~\ref{fig:circuit1} it is the group
of operators to whom
$|\psi_{control}\rangle \otimes |0\rangle \otimes 
|\psi_{target}\rangle$ is an eigenstate, namely 
$\mathbb{I}\otimes Z \otimes \mathbb{I}$ (and trivially
$\mathbb{I}\otimes\mathbb{I}\otimes\mathbb{I}$, which we choose
to ignore as a stabilizer since any three-qubit state
is stabilized by it).

The stabilizer group is always an abelian group and its elements 
 commute, since if:

\begin{equation}
	\label{abelian_stabilizers_equation}
	\forall A,B \in S: AB|\psi\rangle = BA|\psi\rangle = |\psi\rangle
	\Rightarrow [A,B]|\psi\rangle=0
\end{equation}

\subsubsection{Effect of Gates on stabilizers}
In order to determine the effect a Gate operation A has on a
stabilizer, consider the following:

If $S|\psi\rangle = |\psi\rangle$ then:
\begin{equation}
A|\psi\rangle = AS|\psi\rangle = AS\mathbb{I}|\psi\rangle
	= \underbrace{ASA^{\dagger}}_{=S'}A|\psi\rangle
\end{equation}
So we now know that the post-gate state is an Eigenstate of $S'$.

Therefore $S'_{G} = \langle AS_{0}A^{\dagger},...,AS_{n}A^{\dagger}\rangle$.
%This generator expression can be further reduced by taking out
%all $S' \in S'_{G}$ for which $\exists J' \in S'_{G}: \{S',J'\}=0$

\subsubsection{Effect of measurements on stabilizers}
A pauli measurement operator M can either commute with all stabilizer
operators, in which case M itself is a stabilizer already and the
measurement has no effect on the state, or anticommute with at
least one operator in $S_{G}$, since pauli operators as well as
their tensor-products can only commute or anti-commute with each
other.

In that case, to obtain the new stabilizers  $S'_{G}$, proceed
as follows:
\begin{itemize}
	\item Identify $S\in S_{G}: {S,M}=0$
	\item Remove S from $S_G$
	\item Add $M$ to $S_G$ 
	\item for $N \in S_G \cup\overline{X}\cup\overline{Z}:$
		if ${N,M}==0: N=SN$ 
\end{itemize}
where $\overline{X}$ and $\overline{Z}$ are the sets of 
logical Xs and Zs respectively.

\subsubsection{Circuit Analysis in Stabilizer formalism}
After a measurement M, an n qubit input state will always 
collapse into either the +1 or the -1 eigenstate of the 
measurement operator.

In the first case the acting measurement operator becomes 
$\mathbb{I}^{\otimes n}+M$, in the second it becomes
$\mathbb{I}^{\otimes n}-M$. Therefore, in the circuit shown in 
figure~\ref{fig:circuit1}, the measurements are:
\begin{align}
	M_{1} &= \mathbb{I}^{\otimes 3} \pm \mathbb{I}\otimes X
	\otimes X \\
	M_{2} & = \mathbb{I}^{\otimes 3} \pm X \otimes X \otimes
	\mathbb{I} \\
	M_{3} &= \mathbb{I}^{\otimes 3} \pm \mathbb{I} \otimes X
	\otimes \mathbb{I}
\end{align}


\newpage
