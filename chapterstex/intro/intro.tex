In the last few years quantum computers have been the focus of intense research
since they are expected to be able to solve problems that are intractable for
classical computers. 
Quantum computers employ principles of quantum mechanics, whereby states can exist
in superpositions of multiple states, and can be entangled, i.e. correlated
in order to perform computation.

One area where quantum computers are expected to be able to outperform classical
computers is decrypting RSA encryption by efficiently factoring large 
numbers. 
This has recently been shown by researchers at the Beijing
Academy of Quantum Information Sciences to require merely 10 error-corrected
 qubits \cite{beijing} to efficiently factor a 40-bit length number,
and is estimated to require merely 372 error-corrected qubits to efficiently decrypt
2048-bit RSA encryption.
It can therefore with high confidence be said that within the next decade
RSA encryption will no longer be viable for protecting sensitive data.

Others include simulations of quantum systems, which can be of great use in
medical research and quantum chemistry, as well as optimization problems, which
are of great use in logistics and scheduling.
Further, the quantum Fourier transform, which is a quantum algorithm that
can be used to efficiently compute the discrete Fourier transform, can be used for 
things like computing ideal signal output from 5G towers to minimize interference.
While providing significantly less advantage over classical computers than the 
aforementioned applications, the quantum search algorithm also provides a 
square-root improvement in the time complexity of searching for a specific item in 
a database, which could also have wide applications.

In order to be able to use quantum computers for these applications, we need to
ensure their resiliency towards errors introduced by thermal, electromagnetic and 
other noise. 
This can be done via Quantum Error Correction (QEC) codes, which we will introduce
and discuss in this thesis.
