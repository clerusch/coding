\subsection{Decoders for Surface/Toric codes}
The Syndromes on the surface/toric code are a set of nodes and
faces on the code graph. The node ancilla syndromes correspond to
Z errors, while the face ancilla syndromes correspond to X errors.
Since neighboring errors will trigger an ancilla that is between 
both errors twice, a chain of errors will only appear as two ancilla
syndrome bits being flipped at its borders.
The task of a decoding scheme for a surface/toric code is thus to
find the shortest paths between node pairs/face pairs, since the most likely
chain of errors to occur given a $<50\%$ physical error rate is the 
shortest one.\\
In practice, decoders for surface/toric codes only need to be able to
match nodes, since the matching of faces is just matching nodes on the 
dual graphs and the resulting data qubit errors can just be joined
(i.e. if an edge is found to have an error on both the X graph as well as the
dual Z graph, we know a Y error has occurred on that edge/data-qubit).
An example of a distance 5 surface code with two Z errors, one X error and
one Y error is shown in figure \ref{fig: surface_code}.
As with the ringcode, the decoding problem can be seen as either the
solution of equation \ref{eq: pcm} for a minimum weight $\vec{v}_{error}$
or as a graph matching problem.
\subsubsection{MWPM decoding}
\begin{enumerate}
    \item Find a set of unmatched nodes that can be reached from the 
    matching by alternating between matched and unmatched edges. 
    Call these nodes "augmenting nodes".
    \item Find an augmenting path starting from each augmenting node,
    i.e. a path that starts and ends with an unmatched node, and
    alternates between matched and unmatched edges. 
    \item If such a path is found, flip all edges along it from matched
    to unmatched, and vice versa.
    \item Repeat until no augmenting path is found.
\end{enumerate}
This decoding scheme has the advantage of being guaranteed
to find a global optimum of decoding edge paths, i.e. it finds the shortest vector
of edges that are bounded by the syndrome nodes.
Under the assumption of high error rates and/or large decoding 
graphs, this scheme also requires significantly less
computational memory overhead than the union-find scheme
\cite{MWPMDecoder}.
\subsubsection{Union Find decoder}
\begin{enumerate}
    \item Initialize a cluster set for each syndrome node
    \item Grow each cluster by one edge in each direction
    \item Merge all clusters that share a node
    \item For all clusters with an even amount of syndrome nodes,
    perform MWPM within that cluster. Pop the found error edges from
    the graph.
    \item Repeat until all clusters are merged/discarded.
\end{enumerate}
While the union-find decoder is faster for small to medium
sized graphs and relatively simple to implement,
 it is not guaranteed to find a global optimum
and its performance degrades significantly for large
graphs and high error rates \cite{UFDecoder}.
For this reason, a MWPM algorithm was chosen for decoding the toric
subgraphs of the color code in our lifting decoder thresholding
in chapter \ref{sec: lifting}.

\subsection{Color code decoders}
Unlike the surface and toric codes, in the color code the 
data qubits sit on the graphs nodes, and the ancillas on the 
graphs faces. Decoding the color code entails matching 
three differently colored faces to its enclosed nodes.
This is a significantly more challenging task than
decoding the 2D-codes, since three-colored graph matching is a confirmed
NP-hard problem. (reference delfosse paper)
\subsubsection{Lookup table decoding}
This takes a long time to generate and cannot be done to scale,
but is a nice toy.
\subsubsection{Lifting decoder}\label{sec: lifting}
The Lifting decoder works as follows:
\begin{itemize}
    \item Create dual of graph
    \item Generate single-edge-colored subgraphs of the dual
    \item Decode subgraphs using MWPM/Union-Find
    \item Unify all edges from subgraph corrections
    \item Find all shortest-length loops on this union
    \item NOLIFTING CASE ????
    \item All nodes bounded by the faces that are elements of the shortest-length loop sets
    are error nodes. 
\end{itemize}
The reason for this seemingly complicated procedure is that, on its
own, 3-colored matching would be an NP-hard problem. However, by sub-tiling
the graph into smaller subgraphs, we can reduce the problem of decoding
e.g. a honeycomb lattice toric color code to a set of
MWPM-decodable toric graphs that merely need to be "lifted" into a 
combination of subgraph decodings to decode the original color code
graph \cite{delfosse}. 
The polynomial time complexity of the lifting decoder does not
violate the NP-hardness of the 3-color matching problem, since the 
lifting procedure can also fail (NOLIFTING case).
\newpage
\subsection{Thresholds}
To compare different codes and decoding schemes
we introduce the concept of thresholds, whereby the threshold
of a specific code of scalable distance with a specific decoding 
scheme is defined as the physical error rate $per$ at which the logical
error rate becomes greater than $50\%$ in the limit of infinite 
distance. \\
Thresholds can vary depending on the error model, i.e. 
some codes can have a higher threshold for X than for Z errors.
For simplicity's sake in the following, we will assume equal 
X, Y and Z error rates of $\frac{per}{3}$.

Using this error model, we found a threshold of $x\pm y \%$ and
$z\pm q\%$ for the surface and toric code respectively. 
(A BUNCH OF FIGURES AND APPENDIX REFS)

On the color code, we found a threshold of $a\pm b\%$ using the 
lookup table decoder on the Steane code and $c\pm b\%$ using the 
lifting decoder for the scalable hexagonal toric color code. 
(Reference appendix code, include a figure)