\subsubsection{MWPM decoding}
The \emph{Minimum-Weight Perfect Matching} algorithm is a variant of 
Dijkstra's algorithm that can be used to find the shortest vector of edges
that are bounded by the input syndrome nodes.
The algorithm is as follows:
\begin{enumerate}
    \item Find a set of unmatched nodes that can be reached from the 
    Matching\footnote{In graph theory, a matching is a subset of graph
    edges such that no two edges share a common vertex. The Goal of the MWPM
    algorithm is to find a Matching with minimum weight, i.e. a shortest vector of
    edges} by alternating between matched and unmatched edges. 
    Call these nodes "augmenting nodes".
    \item Find an augmenting path starting from each augmenting node,
    i.e. a path that starts and ends with an unmatched node, and
    alternates between matched and unmatched edges. 
    \item If such a path is found, flip all edges along it from matched
    to unmatched, and vice versa.
    \item Repeat until no augmenting path is found.
\end{enumerate}

This decoding scheme has the advantage of being guaranteed
to find a global optimum of decoding edge paths, i.e. it always finds the shortest vector
of edges that are bounded by the syndrome nodes.
Under the assumption of high error rates and/or large decoding 
graphs, this scheme also requires significantly less
computational memory overhead than the union-find scheme
\cite{MWPMDecoder}.