\subsubsection{Lifting decoder}\label{sec: lifting}
The Lifting decoder works as follows:
\begin{itemize}
    \item Create dual of graph
    \item Generate single-edge-colored subgraphs of the dual
    \item Decode subgraphs using MWPM/Union-Find
    \item Unify all edges from subgraph corrections
    \item Find all shortest-length loops on this union
    \item NOLIFTING CASE ????
    \item All nodes bounded by the faces that are elements of the shortest-length loop sets
    are error nodes. 
\end{itemize}
The reason for this seemingly complicated procedure is that, on its
own, 3-colored matching would be an NP-hard problem. However, by sub-tiling
the graph into smaller subgraphs, we can reduce the problem of decoding
e.g. a honeycomb lattice toric color code to a set of
MWPM-decodable toric graphs that merely need to be "lifted" into a 
combination of subgraph decodings to decode the original color code
graph \cite{delfosse}. 
The polynomial time complexity of the lifting decoder does not
violate the NP-hardness of the 3-color matching problem, since the 
lifting procedure can also fail (NOLIFTING case).