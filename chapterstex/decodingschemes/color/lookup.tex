\subsubsection{Lookup table decoding}
A lookup table decoder works by generating the 
syndromes for the entire set of possible input errors, thus creating a 
table holding possible errors responsible for each possible syndrome.
The decoding then consists of merely assuming the minimum weight
error that leads to the known syndrome, since given low physical 
error rate, the least amount of errors leading to an error is the most
probable event.

This decoding scheme is particularly useful for small codes, as well 
as non-topological (random) LDPC (Low-Density-Parity-Check) codes.
A big issue with this decoding scheme is that generating lookup tables is
extremely computationally expensive $(O(2^n))$. This renders it practically
unfeasible to generate lookup tables for codes with a larger number
of total data qubits.

\begin{figure}[h!]
	\begin{center}
	\captionsetup{justification=centering,margin=2cm}
	\includegraphics[scale=0.7]{./img/figures/X7errorlookup.png}\\
	\caption{Lookup table for an X error on the central qubit of
    a Steane code (qubit 7), generating code can be found in Appendix
    \ref{App: lookup_table}}
        
	\label{fig: lookup_table}
	\end{center}
\end{figure}

In figure \ref{fig: lookup_table} is an example of the lookup table result for
an X error on qubit 7 (the central qubit) on the Steane code. 
The resulting syndrome is (1,1,1,0,0,0), with the first three
bits indicating the steane code faces X reaction, and the second three bits
indicating the Steane code faces Z reaction. 
The lookup table will return a set of many possible errors resulting in 
that syndrome, but simply choosing the one with the least number of errors 
(minimum weight) gives the correct error prediction.

Since this is not a distance-scalable code, only a $pseudo$-threshold can
be found here, i.e. the crossing point to worse performance than unencoded
information. As can be seen in figure \ref{fig: steane_threshold}, the
pseudo-threshold is somehow very bad. I do not know why.

\begin{figure}[h!]
	\begin{center}
	\captionsetup{justification=centering,margin=2cm}
	\includegraphics[scale=0.7]{./img/figures/thresholds/steaneLookupThreshold.png}\\
	\caption{Lookup table pseudo threshold for the Steane code, generating code can be found in Appendix
    \ref{App: steane_thresholding}}
        
	\label{fig: steane_threshold}
	\end{center}
\end{figure}
\newpage