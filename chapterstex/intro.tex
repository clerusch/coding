\subsection{Schroedinger and Heisenberg picture}
\subsubsection{Schroedinger picture}
In the Schroedinger picture, we focus on the time evolution of states:
\begin{equation}  
	|\psi\rangle = |\psi\rangle(t) 
\end{equation}
In this picture we can introduce quantum circuit diagram notation, whereby:
\begin{itemize}
	\item States progress in time along horizontal parallel lines
	\item Time goes from left to right
	\item Gates denoted X,Y,Z are the pauli matrices 
		$\sigma_x,\sigma_y,\sigma_z$
	\item Gates can act on one or multiple qubits, whereby an X gate 
		on qubit 1 in a 3-qubit system should be interpreted as:
		\\$(X\otimes \mathbb{I} \otimes \mathbb{I}) (|\psi_1\rangle
		\otimes |\psi_2\rangle \otimes |\psi_3\rangle)$
\end{itemize}
\begin{figure}[h!]
	\begin{center}
\includegraphics[scale=0.5]{img/cnotMeasureCircuit.png}\\
	\caption{A Quantum Circuit, where $|0\rangle$ is the +1
	eigenstate in $\sigma_z$-basis}
	\label{fig:circuit1}
	\end{center}
\end{figure}
\newpage

As can be seen explicitly calculated in the familiar Schroedinger 
picture in Appendix \ref{sec:calc1}, the circuit from figure 
\ref{fig:circuit1} implements a 
CNOT-gate from the control qubit to the target qubit.

We will now analyze this circuit in the Heisenberg picture,
finding that it results in an equivalent result.

\subsubsection{Heisenberg Picture}
In this picture, we focus on the time evolution of operators instead
of states:
\begin{equation}
	A = A(t)
\end{equation}
By considering specifically the operators to which the input state
space is part of those operators eigenstatespace, we can compute
the output of any circuit:
\begin{equation}
	Circuit(|\phi\rangle) = Circuit(A)|\phi\rangle
\end{equation}
if $|\phi\rangle$ is an eigenstate of A.
