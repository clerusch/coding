\subsection{Error Model}
One of the big challenges of physically realizing a quantum 
computer is its subjection to noise in the real world.
Unlike classical computers, the type of error is not limited
to a bitflip, as even single qubit states have an
infinite amount of differing states to it on a Bloch sphere,
and therefore an infinite amount of types of errors can have
occurred in the presence of noise such as thermal or electromagnetic
noise.

Fortunately, this noise can be modeled as a sum of Pauli gates.
Any single qubit error operator matrix E can be written as:
\begin{equation}
    E =
    \left(
    \begin{array}{cc}
        a & b \\
        c & d \\
    \end{array}
    \right) = 
    \alpha \mathbb{I} + \beta X + \delta Y + \gamma Z
\end{equation}
With an apropriate choice of $\alpha, \beta, \gamma, \delta$.
In effect, this means that with probability $\alpha$, the effect of the
error $E|\psi\rangle$ will be $\mathbb{I}$; with probability $\beta$ its effect
will be X, and so on.

Therefore it is sufficient to determine which of these errors $\mathbb{I}$, 
X, Y or Z has occurred, and we can apply the same operator again to return to the 
initial state.
Since an identity noise occurring is irrelevant to us, and XY as
well as ZY (anti-) commute, we need only detect for X and Z
errors occurring in order to detect any single qubit errors. 