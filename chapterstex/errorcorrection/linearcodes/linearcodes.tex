\subsection{Classical codes}
The concept of (classical) error-correcting codes (ECC) was
introduced by Claude Shannon in 1948.
Fundamentally, an ECC encodes $logical$ information within
a large superset of basic information carriers.
In the case of a classical computer, this means encoding a
bitstring within a system containing more physical bits
than the length of the encoded message, with the goal of message transmission
being resilient to some bits being faulty or subject to interference (i.e. EM-interference).

Two such classical ECCs are the repetition and the ring code.
We call these codes linear because their graph representations are
linear, or one-dimensional. In Quantum error correction, we speak of $[[n,k,d]]$ stabilizer
codes if an encoding scheme allows for $n$ physical qubits to 
encode $k$ logical qubits to an error distance of $d$, i.e. $d$ 
arbitrary individual errors being corrigible.
In the following, I will refer to linear or classical codes as having a 
distance of $\frac{1}{2}$, to indicate that they do not protect against an
arbitrary single-qubit error, but only against flips in one specific eigenbasis.
\subsubsection{Repetition code}
For this error code information is encoded by repeating the 
intended message some amount of times, and then decoding it
by performing a majority vote on the transmitted message.


\begin{figure}[h!]
	\begin{center}
	\captionsetup{justification=centering,margin=2cm}
	\includegraphics[scale=0.2]{./img/figures/bitflipSyndromeExtraction3Rep.png}\\
	\caption{Bitflip Syndrome extractor for [[3,1,$\frac{1}{2}$]] repetition code\\
        +1 measurement result on first ancilla indicates a bitflip error
        on qubits 1 or 2, +1 result on second ancilla indicates 
		bitflip on second or third qubit}
	\label{fig: syndrome extractor}
	\end{center}
\end{figure}

A quantum equivalent of the 3-bit repetition code performed on
the message $|1\rangle$ is the [[3,1,$\frac{1}{2}$]] repetition
code depicted in 
figure~\ref{fig: syndrome extractor}
, including so-called
$syndrome\ extraction$. A syndrome is a stabilizer that can be
measured to detect whether and where an error has occurred
in a multi-qubit system. It is crucial that the 
measurement of such syndromes occurs without harming the actual
quantum information stored in the $data-qubits$. Therefore
two additional \emph{ancilla-qubits} (both initialized to 
$|0\rangle$) are attached to the circuit via CNOTs.
This circuit is stabilized by IZZ and ZZI, measured by ancilla 
1 and 2. The measurement result will therefore be a vector of length
two, with each entry either being +1 or -1. To simplify the 
algebra this will be changed to the binary representation of 0 
for +1 and 1 for -1. 

To represent the code, stabilizers can be stacked together to
a so-called parity-check-matrix, which satisfies:
\begin{equation}\label{eq: pcm}
	M_{pc}\cdot \vec{v}_{error} = \vec{v}_{syndrome}
\end{equation}
So e.g. the parity check matrix for the $[3,1,\frac{1}{2}]$
repetition code would be:

\begin{equation}
	M_{pc3} = \left( 
	\begin{array}{ccc}
		1 & 1 & 0 \\
		0 & 1 & 1
	\end{array}
	\right)
\end{equation}
And the syndrome for an X error on the first qubit would be
$\left(\begin{array}{c}1\\0\end{array}\right)$.

If we draw a graph to represent this code, with here square nodes being
 ancilla qubits
and round nodes being data qubits, we obtain the following:

\begin{figure}[h!]
	\begin{center}
	\captionsetup{justification=centering,margin=2cm}
	\includegraphics[scale=0.4]{./img/figures/rep_3_graph.png}\\
	\caption{Graph for [[3,1,$\frac{1}{2}$]] repetition code with error on
    node 1 marked in green and resulting syndrome marked red.
    Squares represent ancilla qubits and circles represent data qubits.}
        
	\label{fig: rep_graph}
	\end{center}
\end{figure}

\newpage
\subsubsection{Ring code}
The ring code's graph essentially simply loops around at the repetition
code's single-edged ancilla nodes via an additional ancilla. 
It's edge matrix where the \emph{n}th row
represents which data qubit is connected to the nth ancilla
qubit is the following:
\begin{equation}
    M_{pc3} = \left(
        \begin{array}{ccc}
            1 & 1 & 0\\
            0 & 1 & 1\\
            1 & 0 & 1\\
        \end{array}
        \right)
\end{equation}

\begin{figure}[h!]
	\begin{center}
	\captionsetup{justification=centering,margin=2cm}
	\includegraphics[scale=0.25]{./img/figures/ring_3_graph.png}\\
	\caption{Graph for [[3,1,$\frac{1}{2}$]] ring code with error on node
    1 marked in green and resulting syndrome marked in red.
    Squares represent ancilla qubits and circles represent data qubits.}
        
	\label{fig: ring_graph}
	\end{center}
\end{figure}

