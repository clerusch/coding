\subsection{2D codes}
Previous research in computer science 
provides a toolset for generating valid codes
from existing encoding schemes. 
Hypergraph product codes, introduced by Tillich and Z\'emor,
of two 
existing codes will always remain a valid detection code.
We can therefore form a hypergraph product code of two repetition
codes for X error detection and Z error detection respectively,
obtaining the [[$d^2$,1,d]]``Surface-Code'' which can detect up
to d of $both$ X and Z errors, and 
therefore any pauli error happening.

The parity check matrix $H$ of a hypergraph product code is generated
by the parity check matrices of two valid codes in the following
way:
\begin{equation}
	H = \left(\begin{array}{cc}
		M_{pcX} & 0 \\
		0 & M_{pcZ} \\
	\end{array}\right)
\end{equation}
Should I introduce graphs/ edge matrices/ hypergraphs first in this section? \\
\\
\\
PUT IN A NICE VISUALISATION OF THE SURFACE CODE.