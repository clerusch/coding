
\subsection{Quantum Error Model}
This way of encoding information however leaves a notable
issue:
It only detects bitflip, or Pauli-X, errors occurring on
the stored quantum information. While using Hadamard gates one
could trivially adapt this code to instead detect Pauli-Z errors,
it is not possible to use linear codes like the repetition code
to $simultaneously$ detect Pauli-X and Pauli-Z errors occurring.

Unlike classical computers, on a quantum computer the type of error
 is not limited
to a bitflip. Even for single-qubit states there exists an
infinite amount of differing possible errors, since when representing a single
qubit state as a vector on a Bloch sphere it immediately becomes apparent
that there are an infinite number of vectors on that sphere which are different
from it. It turns out though, that the change from one normalized
state to another is merely a sum of rotations.

Noise can therefore be modeled as a sum of Pauli gates.
Any single qubit error operator matrix E can be written as:
\begin{equation}
    E =
    \left(
    \begin{array}{cc}
        a & b \\
        c & d \\
    \end{array}
    \right) = 
    \alpha \mathbb{I} + \beta X + \delta Y + \gamma Z
\end{equation}
With an apropriate choice of $\alpha, \beta, \gamma, \delta$.
In effect, this means that with probability $\alpha$, the effect of the
error $E|\psi\rangle$ will be $\mathbb{I}$; with probability $\beta$ its effect
will be X, and so on.

It is hence sufficient to determine which of these errors $\mathbb{I}$, 
X, Y or Z has occurred, and we can apply the appropriate operator to return to the 
initial state.
Since an identity noise occurring is irrelevant to us, and XY as
well as ZY (anti-) commute, we need only detect for X and Z
errors occurring in order to detect any single qubit errors. 
(because of the commutation relation between \{X,Y\} and \{Y,Z\} a Y error
will appear as both an X and Z error).

\subsection{1D/ Linear codes}
From classical computer science there are well known existing
encoding schemes, such as the repetition and ring code.
We call these codes linear because their graph representations are
linear, or one-dimensional. In Quantum error correction, we speak of $[[n,k,d]]$ stabilizer
codes if an encoding scheme allows for $n$ physical qubits to 
encode $k$ logical qubits to an error distance of $d$, i.e. $d$ individual errors 
being corrigible.
In the following I will refer to linear or classical codes as having a 
distance of $\frac{1}{2}$, to indicate that they do not protect against an
arbitrary single qubit error, but only against flips in one specific eigenbasis.
\subsubsection{Repetition code}
For this error code information is encoded by repeating the 
intended message some amount of times, and then decoding it
by performing a majority vote on the transmitted message.


\begin{figure}[h!]
	\begin{center}
	\captionsetup{justification=centering,margin=2cm}
	\includegraphics[scale=0.2]{./img/figures/bitflipSyndromeExtraction3Rep.png}\\
	\caption{Bitflip Syndrome extractor for [[3,1,$\frac{1}{2}$]] repetition code\\
        +1 measurement result on first ancilla indicates a bitflip error
        on qubits 1 or 2, +1 result on second ancilla indicates 
		bitflip on second or third qubit}
	\label{fig: syndrome extractor}
	\end{center}
\end{figure}

A quantum equivalent of the 3-bit repetition code performed on
the message $|1\rangle$ is the [[3,1,$\frac{1}{2}$]] repetition
code depicted in 
figure~\ref{fig: syndrome extractor}
, including so-called
$syndrome\ extraction$. A syndrome is a stabilizer that can be
measured to detect whether and where an error has occurred
in a multi-qubit system. It is crucial that the 
measurement of such syndromes occurs without harming the actual
quantum information stored in the $data-qubits$. Therefore
two additional \emph{ancilla-qubits} (both initialized to 
$|0\rangle$) are attached to the circuit via CNOTs.
This circuit is stabilized by IZZ and ZZI, measured by ancilla 
1 and 2. The measurement result will therefore be a vector of length
two, with each entry either being +1 or -1. To simplify the 
algebra this will be changed to the binary representation of 0 
for +1 and 1 for -1. 

To represent the code, stabilizers can be stacked together to
a so-called parity-check-matrix, which satisfies:
\begin{equation}\label{eq: pcm}
	M_{pc}\cdot \vec{v}_{error} = \vec{v}_{syndrome}
\end{equation}
So e.g. the parity check matrix for the $[3,1,\frac{1}{2}]$
repetition code would be:

\begin{equation}
	M_{pc3} = \left( 
	\begin{array}{ccc}
		1 & 1 & 0 \\
		0 & 1 & 1
	\end{array}
	\right)
\end{equation}
And the syndrome for an X error on the first qubit would be
$\left(\begin{array}{c}1\\0\end{array}\right)$.

If we draw a graph to represent this code, with here square nodes being
 ancilla qubits
and round nodes being data qubits, we obtain the following:

\begin{figure}[h!]
	\begin{center}
	\captionsetup{justification=centering,margin=2cm}
	\includegraphics[scale=0.4]{./img/figures/rep_3_graph.png}\\
	\caption{Graph for [[3,1,$\frac{1}{2}$]] repetition code with error on
    node 1 marked in green and resulting syndrome marked red.
    Squares represent ancilla qubits and circles represent data qubits.}
        
	\label{fig: rep_graph}
	\end{center}
\end{figure}

\subsubsection{Ringcode}
The ring code's graph essentially just loops aroung at the repetition
code's single-edged nodes, so for example its edge matrix or parity-check
matrix for a three-qubit system looks like:
\begin{equation}
    M_{pc3} = \left(
        \begin{array}{ccc}
            1 & 1 & 0\\
            0 & 1 & 1\\
            1 & 0 & 1\\
        \end{array}
        \right)
\end{equation}
\newpage
\subsection{2D codes}
Previous research in computer science 
provides a toolset for generating valid codes
from existing encoding schemes. 
Hypergraph product codes, introduced by Tillich and Z\'emor,
of two 
existing codes will always remain a valid detection code.

The parity check matrix $H$ of a hypergraph product code is generated
by the parity check matrices of two valid codes in the following
% way:
% \begin{equation}
% 	H = \left(\begin{array}{cc}
% 		M_{pcX} & 0 \\
% 		0 & M_{pcZ} \\
% 	\end{array}\right)
% \end{equation}

\subsubsection{Surface code}
We can therefore form a hypergraph product code of two repetition
codes for X error detection and Z error detection respectively,
obtaining the [[$d^2$,1,d]]``Surface-Code'' which can detect up
to d of $both$ X and Z errors, and 
therefore any pauli error happening.
We can draw this code as a graph, whereby the code's stabilizers
are understood as an adjacency Matrix, the data qubits are on 
edges of the graph and the ancilla qubits for Z-checks are on faces
while the ancilla qubits for X-checks lie on nodes.\\
Like the repetition code, the Surface code is a code that is regular until
it's boundary nodes.

INSERT FIGURE SHOWING TORIC CODE

\subsubsection{Toric code}
Similarly, a hypergraph product code of two ring codes can be 
generated. We call this code the "Toric code".\\
INSERT FIGURE SHOWING TORIC CODE

\subsubsection{Color code}
The color code is a code whereby the graph whose adjacency matrix's 
rows are both the code's X stabilizers and Z stabilizers.
Any three-colorable graph represents a valid color code.
On the color code, an error is bounded by syndromic faces of all colors.
\\
INSERT FIGURE SHOWING COLOR CODE

\newpage
\subsection{Decoding schemes}
\subsubsection{Thresholds}
In order to compare different codes and decoding schemes,
we introduce the concept of thresholds.....

\subsubsection{Decoders for Surface/Toric codes}
\subsubsection{MWPM decoding}
\subsubsection{Union Find decoder}

\subsubsection{Color code decoders}

\subsubsection{Lookup table decoding}
This takes a long time to generate and cannot be done to scale,
but is a nice toy.

\subsection{Lifting decoder}
\begin{itemize}
    \item Create dual of graph
    \item Generate single-edge-colored subgraphs of the dual
    \item Decode subgraphs using MWPM/Union-Find
    \item Unify all edges from subgraph corrections
    \item Find all loops on this
\end{itemize}

