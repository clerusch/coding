\subsubsection{Surface code}
We can therefore form a hypergraph product code of two repetition
codes to
obtain the [[$d^2$,1,d]] ``Surface-Code'' which can detect up
to d of $both$ X and Z errors, and 
therefore any error happening \cite{joschka}.
We can draw this code as a graph, whereby the code's stabilizers
are understood as an adjacency Matrix of data to ancilla qubits.
Unlike in figure \ref{fig: surface_code}, it is possible to draw this 
graph without colors, by drawing it such that the data qubits are on 
edges of the graph and the ancilla qubits for Z-checks are on faces
while the ancilla qubits for X-checks lie on nodes.\\
Like the repetition code, the Surface code is a code that is regular until
its boundary nodes. 

\begin{figure}[h!]
	\begin{center}
	\captionsetup{justification=centering,margin=2cm}
	\includegraphics[scale=0.35]{./img/figures/d5surfaceCode.png}\\
	\caption{Distance 5 Surface code with data qubits in white and 
    ancilla qubits in grey. Green Faces tepresent X stabilizers
    and Red faces represent Z stabilizers.
    Errors on data qubits are marked
    by respective Pauli names and violated stabilizers are marked in red.}
	\label{fig: surface_code}
	\end{center}
\end{figure}


INCLUDE LOGICAL OPERATORS
INSERT FIGURE SHOWING SURFACE CODE
\newpage